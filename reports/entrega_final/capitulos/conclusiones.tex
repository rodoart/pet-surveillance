%------------------------------------------------------------------------------%
\section{Conclusiones y trabajo futuro}
%------------------------------------------------------------------------------%
\label{sec:conclusiones}

Se realizó un prototipo funcional que detecta de forma completamente automática mediante la conexión de una cámara usb si un perro o un gato se sube a los muebles. Emite una alerta al usuario, este prototipo se presento a 14 usuarios que contestaron una encuesta con 5 preguntas que nos permiten observar lo siguiente

Se determinó de forma experimental la viabilidad del proyecto: se puede crear una cámara que detecta algunos comportamientos negativos de las mascotas de forma automática, al encuestar usuarios sobre su opinión respecto al funcionamiento del prototipo mediante una demostración estos concordaron en que el prototipo cumple el objetivo y tiene utilidad en su día a día.

Este trabajo presenta una metodología para el desarrollo del proyecto si se decide continuarlo y los costos aproximados que tendría. Incrementarlo en futuras iteraciones será natural bajo este marco de referencia.

Los usuarios de las historias mostraron interés en un sistema que pueda detectar comportamientos negativos de las mascota y aunque algunos les parece útil que verifique que no se suba a algunos muebles, es importante incluir más comportamientos negativos como es la destrucción de objetos.

La validación demostró que el 83\% de las personas encuestadas consideran útil la versión actual del programa y el 75\% además añadió que sería útil que se creara un sistema de notificaciones para cuando ocurran comportamiento negativos.

En versiones futuras del proyecto se mejorará la latencia de la captura de vídeo reduciendo la cantidad de veces que se realizan los cálculos en un intervalo de tiempo dado para alcanzar al menos 24 cuadros por segundo en velocidad de fotogramas y la latencia debe ser menor a 500 ms en tiempo de respuesta.

Se desarrollará una aplicación de PC que se pueda distribuir a los usuarios. Más tarde una versión de Android que se pueda montar en una Raspberry.

Después de revisar el funcionamiento del protitpo los usuarios creen necesaria la introducción de una alerta sonora para la mascota, por lo cual se verificará experimentalmente si se puede utilizar sonido o grabaciones para influir en la conducta de los animales e impedir el comportamiento negativo a distancia.

