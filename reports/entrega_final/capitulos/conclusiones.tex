%------------------------------------------------------------------------------%
\section{Conclusiones y trabajo futuro}
%------------------------------------------------------------------------------%
\label{sec:conclusiones}

Se determinó de forma experimental la viabilidad del proyecto: se puede crear una cámara que detecta algunos comportamientos negativos de las mascotas de forma automática.

Este trabajo presenta una metodología para el desarrollo del proyecto si se decide continuarlo y los costos aproximados que tendría. Incrementarlo en futuras iteraciones será natural bajo este marco de referencia.


Se realizó un prototipo funcional que detecta de forma completamente automática mediante la conexión de una cámara usb si un perro o un gato se sube a los muebles. Emite una alerta al usuario.

Los usuarios de las historias mostraron interés en un sistema que pueda detectar comportamientos negativos de las mascota y aunque algunos les parece útil que verifique que no se suba a algunos muebles, es importante incluir más comportamientos negativos como es la destrucción de objetos.


En versiones futuras del proyecto se mejorará la latencia de la captura de vídeo reduciendo la cantidad de veces que se realizan los cálculos en un intervalo de tiempo dado. 

Se desarrollará una aplicación de PC que se pueda distribuir a los usuarios. Más tarde una versión de Android que se pueda montar en una Raspberry.

Se verificará experimentalmente si se puede castigar con sonido o con grabaciones a los animales e impedir el comportamiento negativo a distancia.

