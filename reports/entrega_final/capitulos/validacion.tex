%------------------------------------------------------------------------------%
\section{Validación}
%------------------------------------------------------------------------------%

\label{sec:validacion}

Como validación técnica se presenta el vídeo que se adjunta en con este trabajo, en el cuál se muestra el programa funcionando de forma adecuada. Se destacan los siguientes hitos:

\begin{itemize}
    \item Identifica correctamente el suelo.
    \item Identifica y crea una frontera correctamente alrededor de la mascota.
    \item Detecta cuando la mascota se sube a un mueble.
    \item La alerta se emite de acuerdo a lo esperado.
\end{itemize}


Para validar el proyecto  en el mercado se decidió tomar la retroalimentación directamente a las personas que participaron en las encuestas del \hyperref[sec:desarrolloconceptual]{desarrrollo conceptual} diseñando una nueva encuesta que incluye preguntas pertinentes al estado de desarrollo del programa. Los resultados están en el \hyperref[ape:anexo_validacion]{anexo B} y se resumen en la gráfica de la \figura{fig:validacion}. 


\begin{figure}
    \centering
    \caption[Validación]{Resultados de la encuesta de validación}
    \label{fig:validacion}
    \includegraphics[width=\linewidth]{validacion.pdf}
  \end{figure}

Las dos métricas cruciales para la validación son el 75\% de las personas respondió que le importaban que las mascotas se suban a sus muebles y que el 
83\% de las personas consideran que la aplicación sería útil en su día a día. Esto demuestra que la utilidad de la aplicación y que presenta valor en el mercado. 


Las otras dos preguntas hablan de próximas iteraciones del desarrollo del proyecto: el 67\% de los usuarios requiere la alarma que espante de forma automática a las mascotas y el 83\% dice que es mejor tener un sistema de notificación, por lo que esta es la dirección que se tomaría, alertar al usuario con una aplicación.