%------------------------------------------------------------------------------%
\section{Introducción}
%------------------------------------------------------------------------------%

En México, 7 de cada 10 hogares mexicanos tienen algún animal doméstico, y de estos, 89\% son perros \parencite{inegi2016}. Una gran parte de estas mascotas pasa su tiempo en solitario en sus viviendas, cuando sus dueños se marchan a realizar sus actividades laborales.

El vínculo que se forma entre una mascota y su amo puede llegar a ser muy profundo. Por esta razón, algunas mascotas y dueños pueden llegar a sentir ansiedad al separase de sus mascotas. En el caso de los perros, esta ansiedad se puede manifestar en comportamientos destructivos hacia los objetos de su hogar \parencite{parthasarathy2006}. 


\subsection{Justificación}
\label{sec:justificacion}

Este trabajo presenta una solución tecnológica para dos de los problemas causados por dejar a las mascotas en solitario:

\begin{enumerate}
  \item Disminuir la ansiedad de los dueños al dejar a su mascota sin vigilancia.
  \item Reducir o impedir el comportamiento destructivo cuando la mascota está en solitario. 
\end{enumerate}

Esta problemática se origina en las exigencias de los horarios laborales, en los que una persona promedio debe dedicar hasta 40 horas a la
semana de tiempo en la oficina. La ansiedad causada por la falta de atención y la soledad es la causa de los destrozos y malos comportamientos por parte de los animales \parencite{ibanes2022}.

Una solución que ha surgido para paliar este problema es el uso de paseadores. Sin embargo, el tiempo del paseo por lo regular no es mayor a una hora, por día. Insuficiente para la enorme cantidad de tiempo que los animales pasan en soledad día a día y además esto sólo es para los perros, ignorando a todo el grupo de los felinos.


\subsection{Objetivos}

De manera resumida, el presente trabajo plantea el uso de un sistema de visión artificial para monitorear la actividad de mascotas en tiempo real y reforzar ciertas pautas de entrenamiento, principalmente evitar la destrucción de mobiliario y el uso de este por parte de los animales. En la  \hyperref[sec:objetivos]{sección de objetivos se desarrollan más esta idea.}

\subsection{Estructura}

A continuación se hace una breve descripción de las secciones de este trabajo.

\begin{itemize}
  \item \hyperref[sec:objetivos]{Objetivos}. Se define cuál es el alcance del proyecto y los hitos que lo conforman. 
  \item \hyperref[sec:desarrolloconceptual]{Desarrollo conceptual}. Detalla el procedimiento y metodologías utilizadas para generar las ideas para el desarrollo del proyecto y cómo se llegó a la idea de la forma final del prototipo: un sistema que detecte el comportamiento simple de subirse a los muebles, que se pueda expandir a futuros comportamientos negativos.
  \item \hyperref[sec:metodologia]{Metodología}. Se describe la planificación y la metodología ágil que se usó a lo largo del proyecto SCRUM+LEAN+KANBAN. Los roles dentro del proyecto y las historias iniciales.
  \item \hyperref[sec:implementacion]{Implementación}. Cómo fueron los \textit{sprints} iniciales del proyecto y una descripción breve del desarrollo técnico. Además contiene posibles soluciones a contingencias.
  \item \hyperref[sec:validacion]{Validación}. Se mostró al usuario el resultado final del proyecto mediante una cámara usb conectada a una computadora y se preguntaron sus reflexiones acerca del sistema propuesto.
  \item \hyperref[sec:conclusiones]{Conclusiones y Trabajo Futuro}. Definen que se ha logrado con el proyecto y las futuras mejoras.
  \item \hyperref[ape:tablaencuesta]{Anexos}.
\end{itemize}


