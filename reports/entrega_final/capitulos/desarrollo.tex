%------------------------------------------------------------------------------%
\section{Desarollo conceptual}
%------------------------------------------------------------------------------%
\label{sec:desarrolloconceptual}
Durante el desarrollo conceptual de este proyecto, se utilizó la metodología \textit{Design Thinking}, propuesto inicialmente por Larry Leifer \textcite{plattner2011}. Con el objetivo de obtener un proyecto bien planteado, basado en la retroalimentación de los usuarios finales se divide en las siguientes etapas:


\begin{itemize}
  \item Empatizar
  \item Definir
  \item Investigación de antecedentes
  \item Idear
  \item Prototipar: Cual será el prototipo que vas a utilizar
  \item Selección de prototipos y criterio para hacerlo  
\end{itemize}


%------------------------------------------------------------------------------%
\subsection{Empatizar}

Es la etapa inicial y se considera la más importante, pues las demás etapas se basan en esta. Se requiere conocer al cliente abordando sus verdaderas necesidades y haciéndolas propias. 

En el caso de este proyecto, esta etapa se realizó mediante la aplicación de un cuestionario a familiares, amigos y conocidos. Las preguntas ahondaban en el estado socioeconómica, la relación de los dueños con sus mascotas y sus ideas concretas sobre el problema planteado en la \hyperref[sec:justificacion]{justificación} del problema. Un resumen de las respuestas orginales se puede encontrar en el \hyperref[ape:tablaencuesta]{apéndice A}.


Se deduce de la encuesta que el usuario promedio de la aplicación son personas de entre 25 y 35 años con entre una y dos mascotas, con un salario entre 15,000 y 35,000 pesos mexicanos, viven en departamentos o casas sin patio, adicionalmente trabajan 40 horas a la semana y realizan actividades fuera de su casa lo que les impide cuidar a su mascota durante gran parte del día. Así mismo el usuario tiene la capacidad de acceder a una cámara web desde su laptop o computadora y deben ser nativos digitales cómodos con el uso de aplicaciones web.

Así mismo el usuario ve a las mascotas como parte importante de su núcleo familiar, no como un elemento de guardia o de servicio. Parte de ver a sus mascotas como parte de su núcleo familiar genera preocupación al dejarlo sin supervisión, especialmente por situaciones que lo pudieran poner en peligro, como el que tire algún objeto y se lastime, independientemente del costo material que pudiera ocasionar.

Con los datos de la encuesta (\hyperref[ape:tablaencuesta]{apéndice A}) también se realizó el mapa de empatía de la \figura{fig:mapaempatia}. 


\begin{landscape}
  \begin{figure}
      \centering
      \caption[Mapa de empatía del proyecto.]{Mapa de empatía realizado con los resultados. Cada sección de la figura corresponde a distintas experiencias sensoriales y emocionales acerca del problema a trata. Los esfuerzos representa lo que el usuario final está dispuesto a hacer para resolver el problema y los resultados lo que espera lograr.}
      \label{fig:mapaempatia}
      \includegraphics[width=\linewidth]{mapa_empatia.pdf}
  \end{figure}
\end{landscape}

%------------------------------------------------------------------------------%
\subsection{Definir}
\label{sec:definir}

En esta etapa se identificaron las necesidades principales. Luego de la encuesta y el mapa de empatía (\figura{fig:mapaempatia}) es fácil ver que las necesidades principales son:


\begin{itemize}
  \item Una manera de mantener a la mascota vigilada para que no se haga daño.\item Disminuir el comportamiento destructivo cuando la mascota abandona la casa.
\end{itemize}

%------------------------------------------------------------------------------%
\subsection{Idear}
\label{sec:idear}

Consiste en pensar en distintas soluciones a los problemas de la \hyperref[sec:definir]{etapa de definición}.

Se llegó al conceso de que una cámara, conectada a una aplicación que transmita en tiempo real a las mascotas, es la solución a los problemas de vigilancia. Para el problema de disminuir el comportamiento destructivo se pensó en distintas soluciones que analicen el video en tiempo real y de forma automática, notifiquen y/o emitan una alerta al dueño de la mascota cuando esta :

\begin{itemize}
  \item está arriba de los muebles o en zonas indebidas.
  \item destruye objetos, como calzado o sandalias.
  \item está ladrando o aullando en exceso.
  \item se pelea con otras mascotas.
\end{itemize}

Luego de cierta deliberación, se determinó que el la detección automática de  comportamiento negativo más factible a programar dentro de las limitaciones de tiempo es el problema de cuando la mascota está en zonas indebidas, concretamente arriba de muebles. Los otros problemas quedarán pendientes para futuras iteraciones de este proyecto.

%------------------------------------------------------------------------------%
\subsection{Prototipar}

En esta etapa, se realiza un prototipo de la solución, que se mostrará más tarde a los encuestados.


Realizar una aplicación que envía video real a través de la nube no es novedoso. Detectar comportamientos inusuales de forma automática en las mascotas sí. Es por eso que se decidió que para el prototipo lo primero que se tenía que pensar es cómo ésto era factible. 

Se decidió que para el alcance del proyecto lo ideal era identificar sólo un comportamiento negativo. Luego de una investigación exhaustiva y cualitativa en Google y Youtube se verificó que una gran parte de los problemas de mal comportamiento en mascotas es la destrucción de muebles. En  una gran parte de estos vídeos la mascota destruye el mueble desde arriba del mismo. Si se detecta cuando la mascota se sube a un mueble, se puede evitar la destrucción en una buena parte de los casos. Se propone un programa que actúe como el de \figura{fig:mascotas}.


\begin{figure}
  \centering
  \caption[Proceso de identificación ]{Proceso de detección de la mascota cuando está sobre los muebles en tiempo real:  La entrada (en la parte superior) es una secuencia de vídeo. Esta secuencia en video puede ser un fragmento de unos cuantos cuadros del vídeo en tiempo real.  Dos procesos ocurren de forma semi simultánea. \textit{Procesar el fondo}: Luego de varios cuadros estáticos, se determinan las partes de la imagen que no están en movimiento. Con una red de segmentación automática, se segmenta la imgagen y se detecta el suelo. \textit{Procesar objetos en movimiento}: Se contrastan los objetos que difieren del fondo, se detectan sus fronteras, y se determina si se trata de una mascota o no. Se detecta la probabilidad de estar fuera del suelo. Si esta probabilidad es alta por varios segundos, se emite la alerta.}
  \label{fig:mascotas}
  \includegraphics[width=\columnwidth]{mascotas.pdf}
\end{figure}


A partir del diagrama del proceso de detección, se realizó una animación de cómo funcionaría la aplicación en su versión final (\figura{fig:animacion}).

\begin{figure}
  \centering
  \caption[Prototipo del proyecto final.]{Protipo del proyecto final. Requiere \textit{Acrobat Reader}, \textit{KDE Okular}, \textit{PDF-XChange} o \textit{Foxit Reader} para  visualizar, de lo contrario puede verse en el archivo \textit{.gif} adjunto a este documento.}
  \animategraphics[controls,autoplay,width=0.5\columnwidth]{12}{gif/e05a1b6f887641509ed0d893e0d9609aEX3IJwYFFaSk4UUz-}{0}{84}
  \label{fig:animacion}
\end{figure}


%------------------------------------------------------------------------------%
\subsection{Evaluar}

El video prototipo de la \figura{fig:animacion} fue mostrado a los usuarios que habían respondido las preguntas de la primera encuesta. Los comentarios mostrados fueron positivos, destacando que esta función de detección de animales sobre los muebles ``sería de gran utilidad'' y ``estarían dispuestos a pagar por ella.''. Algunos sugirieron añadir características como conexión con servicios de paseo o cuidado de mascotas, monitoreo de estrés, monitoreo de ejercicio. Estas soluciones se tendrán en cuenta para mejoras futuras.