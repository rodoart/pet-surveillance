%------------------------------------------------------------------------------%
\section{Objetivos}
%------------------------------------------------------------------------------%
\label{sec:objetivos}


El objetivo del trabajo es:

\begin{quotebox}
  Probar que un sistema de vigilancia de mascotas que identifique de forma automática algunos comportamientos negativos es una propuesta viable.
\end{quotebox}


A partir de este objetivo podemos definir los siguientes objetivos específicos:

\begin{itemize}
  \item Determinar quienes serán los usuarios finales que tienen interés en la propuesta.
  \item Generar historias con los usuarios interesados y consultarlos a lo largo del proyecto
  \item Encontrar cuáles son las características más deseables por los posibles usuarios.
  \item Generar un plan de trabajo para realizar el proyecto.
  \item Generar un prototipo con alguna de estas características.
  \item Presentar el prototipo y verificar si los usuarios están satisfechos con el resultado.
\end{itemize}


% Esta sección es el puente entre el estudio del dominio y la contribución a realizar. 

% Hay que definir el Objetivo Geneal (resumido en un par de líneas y explicar el qué, para qué y cómo se va a desarrollar la propuesta) y los Objetivos Específicos (suelen ser explicaciones de los diferentes pasos a seguir en la consecución del objetivo general)

% Para realizar este apartado, podrás apoyarte en los principios MELDS.