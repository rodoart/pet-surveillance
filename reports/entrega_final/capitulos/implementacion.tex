%------------------------------------------------------------------------------%
\section{Implementación de la propuesta}
%------------------------------------------------------------------------------%
\label{sec:implementacion}

\subsection{Planificación y estimación}

En la \tabla{tab:sprints} se puede apreciar a detalle la planificación del proyecto en \textit{sprints}. En la \tabla{tab:costos} se muestran los costes asociados a las distintas actividades y materiales requeridos durante el mismo.


\begin{table}
	\small\centering
    \caption{Cronograma y planificación de los \textit{sprints}. La ``S'' es de \textit{sprint}.}
    \label{tab:sprints}
    \begin{NiceTabular}{p{4.5cm}|c|c|c|c|c|c|c|c|c|c|c|c|c}[
        code-before = 
        \cellcolor{mygreen}{2-2, 2-3, 2-8, 2-11}
        \cellcolor{mygreen}{3-2, 3-3, 3-8, 3-11}
        \rectanglecolor{mygreen}{4-4}{4-6}
        \rectanglecolor{mygreen}{5-4}{5-7}\rectanglecolor{mygreen}{5-9}{5-13}
        \cellcolor{mygreen}{6-8, 6-11}
        \rectanglecolor{mygreen}{7-4}{7-13}
        \rectanglecolor{mygreen}{8-7}{8-13}
        \cellcolor{mygreen}{9-7, 9-10, 9-13}
     ]
        \hline
        Actividad                  & S1& S2& S3& S4& S5& S6& S7& S8& S9&S10&S11&S12\\
        \hline
        Investigación del producto &   &   &   &   &   &   &   &   &   &   &   &   \\
        \hline
        Diseño                     &   &   &   &   &   &   &   &   &   &   &   &   \\
        \hline
        Desarrollo red neuronal    &   &   &   &   &   &   &   &   &   &   &   &   \\
        \hline
        Desarrollo aplicación web  &   &   &   &   &   &   &   &   &   &   &   &   \\
        \hline
        Retroalimentación          &   &   &   &   &   &   &   &   &   &   &   &   \\
        \hline
        Desarrollo 
        API             &   &   &   &   &   &   &   &   &   &   &   &   \\
        \hline
        Entrenamiento y mejora red neuronal &   &   &   &   &   &   &   &   &   &   &   \\
        \hline
        Lanzamiento PMV            &   &   &   &   &   &   &   &   &   &   &   &   \\
        \hline      
    \end{NiceTabular}
\end{table}



\begin{table}
    \small\centering
    \caption{Costos del proyecto.}
    \label{tab:costos}
    \begin{tabular}{p{4.5cm}rrrr}
        \toprule
        Recursos Humanos & Cantidad & Costo mensual & Meses & Subtotal \\
        \midrule
        \textit{Senior FullStack Developer} & 1 & 50,000 & 3 & 150,000 \\
        \textit{Junior Fullstack Developer} & 1 & 28,000 & 3 & 84,000 \\
        \textit{Senior Ux Design}er & 1 & 35,000 & 3 & 105,000 \\
        Arquitecto de sistemas \textit{Cloud mid level} & 1 & 75,000 & 3 & 225,000 \\
        \textit{Product Owner }& 1 & 65,000 & 3 & 195,000 \\
        \midrule
        &  &  &  Total & 759,000 \\
        \bottomrule
        \toprule
        Servicios & Cantidad & Costo mensual & Meses & Subtotal \\
        \midrule
        Servicios Azure & 1 & 51,413 & 3 & 154,240 \\
        \midrule
        &  &  &  Total & 154,240 \\
        \bottomrule
        \toprule 
        Materiales & & Cantidad & Costo & Subtotal \\
        \midrule
        \textit{Laptop hp zbook 14 Firefly G8} & & 4 & 35,000 & 140,000 \\
        \textit{Macbook Pro MKGP3E/A} & & 2 & 49,000 & 98,000 \\
        \midrule
        &  &  Total &  & 238,000 \\
        \bottomrule 
        \toprule 
        Infraestructura &  & Costo mensual & Meses & Subtotal \\
        \midrule
        Internet & & 3,000 & 14 & 42,000 \\
        \midrule
        &  &  Total &  & 42,000 \\
        \bottomrule
        \toprule 
        \multicolumn{4}{r}{\textbf{Total global del proyecto}} & \textbf{1,193,240} \\
        \bottomrule
    \end{tabular}
\end{table}

\subsection{Prototipo}

El prototipo del proyecto se puede encontrar en el repositorio de:

\begin{quotebox}
    \href{https://github.com/rodoart/pet-surveillance}{https://github.com/rodoart/pet-surveillance}
\end{quotebox}    

El prototipo sigue el diagrama de la \figura{fig:funcionamiento}. 

\subsubsection{Descripción técnica}

De manera técnica el prototipo sigue la  \figura{fig:mascotas}. A continuación haremos una descripción breve de cómo su programación y funcionamiento.

Los requisitos para poder ejecutar el prototipo son:

\begin{itemize}
    \item Python 3.7
    \item TensorFlow
    \item Keras, para la red que detecta a las mascotas, las etiqueta, y genera cajas.
    \item Torch, para la red que realiza la segmentación semántica de la habitación.
    \item OpenCV2, para el procesamiento de imágenes y vídeo.
    \item GitHub, para el manejo de versiones.
    \item DVC, para el manejo de conjuntos de datos y archivos no binarios.
\end{itemize}

El programa puede cargar un vídeo desde un archivo de formato común —.mp4, .mpeg, etc— o conectarse a una cámara usb. Tanto la cámara como el vídeo deben ser grabados de forma estática, horizontal, con el suelo bien descubierto y ocupando al menos una quinta parte del cuadro. El marco de trabajo que se uso para esta funcionalidad se obtuvo de \textcite{juras2018}.

En la \figura{fig:mascotas} se aprecia que el programa tiene dos procesos principales: detectar el suelo y detectar y encuadrar a la mascota. 

El primer proceso —detectar el suelo— se realiza tomando el promedio de los primeros cuadros del vídeo, y obteniendo una imagen de fondo en la cual se realiza segmentación semántica —división en zonas por características comunes— utilizando una red neurona convolucional doble pre-entrenada. Para este proyecto se utilizó la de \textcite{aung2022} que tiene la opción de utilizar múltiples modelos pre-entrenados con distintas bases de datos.

El resultado de la segmentación semántica se aprecia en \figura{fig:segmentacion}, una imagen separada por regiones con características similares representadas por un color uniforme. Estas regiones no están etiquetadas, por lo que se determina cual corresponde al suelo mediante estadística, bajo las hipótesis de que una región que es grande, que comparte mucho borde con el borde inferior de la imagen  y que es aquella que tiene su centroide más bajo. 

\begin{figure}
    \centering
    \caption[Ejemplo de segmentación.]{Ejemplo de imagen segmentada: a la \textit{izquierda} la imagen original, a la \textit{derecha} la imagen segmentada.} 
    \label{fig:segmentacion}
    \includegraphics[width=\linewidth]{semantic_segmentation_example.png}
    \source{La imagen original es del \textit{dataset} \figurecite{unity2022} y  la imagen procesada de procesó usando \figurecite{unity2022}}
\end{figure}

El segundo proceso es detectar, etiquetar y delimitar a los animales en una frontera rectangular. Esto se ha realizado con las librerías de detección de objetos de \textcite{tensorflow2022}.


Los proceso convergen para identificar cuando el animal se está subiendo a los muebles. Se mide que tanta fracción del rectángulo del animal interseca el suelo, si esta es menor que el 15\% durante aproximadamente 5s, se considera que el perro ya no está en el suelo y debería estar en los muebles. Emitiendo una alarma, que en el prototipo se trata sólo de un mensaje visual, pero en futuras versiones debería emitir un sonido molesto para el animal.


\begin{figure}
    \centering
    \caption[Diagrama del funcionamiento de la aplicación.]{Diagrama del funcionamiento de la aplicación.}
    \label{fig:funcionamiento}
    \input{../figures/prototipo.tikz}
\end{figure}


\subsection{Despliegue}

El despliegue se dará en los \textit{sprint} 6,9 y 12, como está demostrado en la tabla (planificación), esto al generar una actualización del sitio web usando Azure VM's para la liberación, actualizando la aplicación. 

\subsubsection{Configuración de entornos de desarrollo y producción}

Al usar Azure, específicamente su modalidad de \textit{IaaS} —\textit{Infrastructure As A Service} o Infrastructura como servicio— podemos aprovechar las facilidades que otorga al poder usar diferentes entornos para distintas situaciones. 

Se configura un grupo de recursos —\textit{Resource Group}— en Azure para el entorno de producción y uno para el entorno de desarrollo, usando Azure Virtual Desktop los desarrolladores pueden tener acceso a una máquina virtual preconfigurada con las aplicaciones necesarias, desde allí ocuparan visual studio para subir su código a GitHub, una vez publicado en GitHub usando Azure pipelines se automatizará el proceso de compilación y publicación, este proceso da como resultado artefactos con los archivos necesarios para ejecutar la aplicación que automáticamente son publicados en el entorno de desarrollo y producción.

\subsection{Posibles contingencias}

En esta sección se describen las principales riesgos al proyecto, una descripción en breve del plan para mitigar o resolver en caso de ocurrencia.

\subsubsection{Agotar el presupuesto}

Debido a la metodología ágil del proyecto. Será realizado por etapas, en las que se entregarán versiones funcionales del sistema. Lo que hace que el riesgo sea menor, debido a que el proyecto puede llevarse a producción, o mostrarse como una versión de prueba en la que se pidan donativos en organizaciones de microfinanciamiento como Patreon.


\subsubsection{No encontrar un nicho en el mercado}

Incluso teniendo la retroalimentación de los clientes, puede haber peligros en esta área, el producto inicial podría ser demasiado costoso, o los potenciales clientes podrían ser menos de los que se están esperando. Este es el mayor riesgo al proyecto, se tendría que replantear una alternativa con la tecnología desarrollada, por ejemplo, el proyecto fácilmente puede ser repensado como un sistema de vigilancia automática o como una aplicación que identifica los rostros de las mascotas y les aplica filtros y fondos en un ambiente hogareño.



\subsection{Mantenimiento}

El plan de mantenimiento contempla un desarrollo continuo de la aplicación en concordancia con el modelo SCRUM, tanto para mantener la funcionalidad existente como para desarrollar funcionalidades nuevas que sumen a la aplicación en sí misma.