\documentclass[12pt,a4paper]{proyectoinnovacion}
%------------------------------------------------------------------------------%
% INFORMACIÓN DEL ARTÍCULO Y METADATOS                                         %
%------------------------------------------------------------------------------%
\firstauthor{Rodolfo Arturo González Trillo}
\secondauthor{Carlos Oswaldo Alfaro Rodríguez}
\thirdauthor{}

\university{Universidad Internacional de La Rioja}
\school{Escuela Superior de Ingeniería y Tecnología}
\master{Maestría en Inteligencia Artificial}
\title{Sistema de visión artificial para monitoreo de mascotas}
\keywords{ia, unir, masctoas, monitoreo}
\date{\today}
%------------------------------------------------------------------------------%

\usepackage{pdflscape}
\usepackage{animate}


%%%%%%%%%%%%%%%%%%%%%%%%%%%%%%%%%%%%%%%%%%%%%%%%%%%%%%%%%%%%%%%%%%%%%%%%%%%%%%%%
\begin{document}

%------------------------------------------------------------------------------%
% Página de Título                                                             %
%------------------------------------------------------------------------------%
\maketitle

%------------------------------------------------------------------------------%
% Índice de contenidos                                                         %
%------------------------------------------------------------------------------%
{
  \setcounter{page}{2} 
  \hypersetup{linkcolor=black}
  \tableofcontents\thispagestyle{fancy}
}
\pagebreak

%------------------------------------------------------------------------------%
\section{Introducción}
%------------------------------------------------------------------------------%

En México, 7 de cada 10 hogares mexicanos tienen algún animal doméstico, y de estos, 89\% son perros \parencite{inegi2016}. Una gran parte de estas mascotas pasa su tiempo en solitario en sus viviendas, cuando sus dueños se marchan a realizar sus activades laborales.

El vínculo que se forma entre una mascota y su amo puede llegar a ser muy profundo. Por esta razón, algunas mascotas y dueños pueden llegar a sentir ansiedad al separse de sus mascotas. En el caso de los perros, esta ansiedad se puede manifestar en comportamientos destructivos hacia los objetos de su hogar \parencite{parthasarathy2006}. 


\subsection{Justificación}
\label{sec:justificacion}

Este trabajo presenta una solución tecnológica para dos de los problemas causados por dejar a las mascotas en solitario:

\begin{enumerate}
  \item Disminuir la ansiedad de los dueños al dejar a su mascota sin vigilancia.
  \item Reducir o impedir el comportamiento destructivo cuando la mascota está en solitario. 
\end{enumerate}

Esta problemática se origina en las exigencias de los horarios laborales, en los que una persona promeido debe dedicar hasta 40 horas a la
semana de tiempo en la oficina. La ansiedad causada por la falta de atención y la soledad es la causa de los destrozos y malos comportamientos por parte de los animales \parencite{ibanes2022}.

Una solución que ha surgido para paliar este problema es el uso de paseadores. Sin embargo, el tiempo del paseo por lo regular no es mayor a una hora, por día. Insuficiente para la enorme cantidad de tiempo que los animales pasan en soledad día a día y además esto sólo es para los perros, ignorando a todo el grupo de los felinos.


\subsection{Objetivos}

El presente producto plantea el uso de un sistema de visión artificial para monitorear la actividad de mascotas en tiempo real y reforzar ciertas pautas de entrenamiento, principalmente evitar la destrucción de mobiliario y el uso de este por parte de los animales. 

\subsection{Estructura}

A continuación se hace una breve descripción de las secciones de este trabajo.

\begin{itemize}
  \item Objetivos.
  \item Desarollo conceptual.
\end{itemize}



%------------------------------------------------------------------------------%
\section{Objetivos}
%------------------------------------------------------------------------------%

El objetivo del trabajo es:

\begin{quotebox}
  Probar que un sistema de vigilancia de mascotas que identifique de forma automática algunos comportamientos negativos es una propuesta viable.
\end{quotebox}


% Esta sección es el puente entre el estudio del dominio y la contribución a realizar. 

% Hay que definir el Objetivo Geneal (resumido en un par de líneas y explicar el qué, para qué y cómo se va a desarrollar la propuesta) y los Objetivos Específicos (suelen ser explicaciones de los diferentes pasos a seguir en la consecución del objetivo general)

% Para realizar este apartado, podrás apoyarte en los principios MELDS.


%------------------------------------------------------------------------------%
\section{Desarollo conceptual}
%------------------------------------------------------------------------------%

Durante el desarrollo conceptual de este proyecto, se utilizó la metodología \textit{Design Thinking}, propuesto inicialmente por Larry Leifer \textcite{plattner2011}. Con el objetivo de obtener un proyecto bien planteado, basado en la retroalimentación de los usuarios finales se divide en las siguientes etapas:


\begin{itemize}
  \item Empatizar
  \item Definir
  \item Investigación de antecedentes
  \item Idear
  \item Prototipar: Cual será el prototipo que vas a utilizar
  \item Selección de prototipos y criterio para hacerlo  
\end{itemize}


%------------------------------------------------------------------------------%
\subsection{Empatizar}

Es la etapa inicial y se considera la más importante, pues las demás etapas se basan en esta. Se requiere conocer al cliente abordando sus verdaderas necesidades y haciéndolas propias. 

En el caso de este proyecto, esta etapa se realizó mediante la aplicación de un cuestionario a familiares, amigos y conocidos. Las preguntas ahondaban en el estado socioeconómica, la relación de los dueños con sus mascotas y sus ideas concretas sobre el problema planteado en la \hyperref[sec:justificacion]{justificación} del problema. Un resumen de las respuestas orginales se puede encontrar en una hoja de cálculo, anexa a este documento. %Probablemente se deba pasar a los apéndices.


Se deduce de la encuesta que el usuario promedio de la aplicación son personas de entre 25 y 35 años con entre una y dos mascotas, con un salario entre 15,000 y 35,000 pesos mexicanos, viven en departamentos o casas sin patio, adicionalmente trabajan 40 horas a la semana y realizan actividades fuera de su casa lo que les impide cuidar a su mascota durante gran parte del día. Así mismo el usuario tiene la capacidad de acceder a una cámara web desde su laptop o computadora y deben ser nativos digitales cómodos con el uso de aplicaciones web.

Así mismo el usuario ve a las mascotas como parte importante de su núcleo familiar, no como un elemento de guardia o de servicio. Parte de ver a sus mascotas como parte de su núcleo familiar genera preocupación al dejarlo sin supervisión, especialmente por situaciones que lo pudieran poner en peligro, como el que tire algún objeto y se lastime, independientemente del costo material que pudiera ocasionar.

Con los datos de la encuesta también se realizó el mapa de empatía de la \figura{fig:mapaempatia}. 


\begin{landscape}
  \begin{figure}
      \label{fig:mapaempatia}
      \centering
      \caption[Mapa de empatía del proyecto.]{Mapa de empatía realizado con los resultados. Cada sección de la figura corresponde a distintas experiencias sensoriales y emocionales acerca del problema a trata. Los esfuerzos representa lo que el usuario final está dispuesto a hacer para resolver el problema y los resultados lo que espera lograr.}
      \includegraphics[width=\linewidth]{mapa_empatia.pdf}
  \end{figure}
\end{landscape}

%------------------------------------------------------------------------------%
\subsection{Definir}
\label{sec:definir}

En esta etapa se identificaron las necesidades principales. Luego de la encuesta y el mapa de empatía (\figura{fig:mapaempatia}) es fácil ver que las necesidades principales son:


\begin{itemize}
  \item Una manera de mantener a la mascota vigilada para que no se haga daño.\item Disminuir el comportamiento destructivo cuando la mascota abandona la casa.
\end{itemize}

%------------------------------------------------------------------------------%
\subsection{Idear}
\label{sec:idear}

Consiste en pensar en distintas soluciones a los problemas de la \hyperref[sec:definir]{etapa de definición}.

Se llegó al conceso de que una cámara, conectada a una aplicación que transmita en tiempo real a las mascotas, es la solución a los problemas de vigilancia. Para el problema de disminuir el comportamiento destructivo se pensó en distintas soluciones que analicen el video en tiempo real y de forma automática, notifiquen y/o emitan una alerta al dueño de la mascota cuando esta :

\begin{itemize}
  \item está arriba de los muebles o en zonas indebidas.
  \item destruye objetos, como calzado o sandalias.
  \item está ladrando o aullando en exceso.
  \item se pelea con otras mascotas.
\end{itemize}

Luego de cierta deliberación, se determinó que el la detección automática de  comportamiento negativo más factible a programar dentro de las limitaciones de tiempo es el problema de cuando la mascota está en zonas indebidas, concretamente arriba de muebles. Los otros problemas quedarán pendientes para futuras iteraciones de este proyecto.

%------------------------------------------------------------------------------%
\subsection{Prototipar}

En esta etapa, se realiza un prototipo de la solución, que se mostrará más tarde a los encuestados.

Realizar una aplicación que envía video real a través de la nube no es novedoso. Detectar comportamientos inusuales de forma automática en las mascotas sí. Es por eso que se decidió que para el prototipo lo primero que se tenía que pensar es cómo ésto era factible. 

Recordando que el comportamiento que se quiere resolver en este trabajo es cuando las mascotas se suben a los muebles, la solución ideada para un función que detecte esto está en la \figura{fig:mascotas}.


\begin{figure}
  \centering
  \caption[Proceso de identificación ]{Proceso de detección de la mascota cuando está sobre los muebles en tiempo real:  La entrada (en la parte superior) es una secuencia de vídeo. Esta secuencia en video puede ser un fragmento de unos cuantos cuadros del vídeo en tiempo real.  Dos procesos ocurren de forma semi simultánea. \textit{Procesar el fondo}: Luego de varios cuadros estáticos, se determinan las partes de la imagen que no están en movimiento. Con una red de segmentación automática, se segmenta la imgagen y se detecta el suelo. \textit{Procesar objetos en movimiento}: Se contrastan los objetos que difieren del fondo, se detectan sus fronteras, y se determina si se trata de una mascota o no. Se detecta la probabilidad de estar fuera del suelo. Si esta probabilidad es alta por varios segundos, se emite la alerta.}
  \label{fig:mascotas}
  \includegraphics[width=\columnwidth]{mascotas.pdf}
\end{figure}


A partir del diagrama del proceso de detección, se realizó una animación de cómo funcionaría la aplicación en su versión final (\figura{fig:animacion}).

\begin{figure}
  \centering
  \caption[Prototipo del proyecto final.]{Protipo del proyecto final. Requiere \textit{Acrobat Reader}, \textit{KDE Okular}, \textit{PDF-XChange} o \textit{Foxit Reader} para  visualizar, de lo contrario puede verse en el archivo \textit{.gif} adjunto a este documento.}
  \animategraphics[controls,autoplay,width=0.5\columnwidth]{12}{gif/e05a1b6f887641509ed0d893e0d9609aEX3IJwYFFaSk4UUz-}{0}{84}
  \label{fig:animacion}
\end{figure}


%------------------------------------------------------------------------------%
\subsection{Evaluar}

El video prototipo de la \figura{fig:animacion} fue mostrado a los usuarios que habían respondido las preguntas de la primera encuesta. Los comentarios mostrados fueron positivos, destacando que esta función de detección de animales sobre los muebles ``sería de gran utilidad'' y ``estarían dispuestos a pagar por ella.''. Algunos sugirieron añadir características como conexión con servicios de paseo o cuidado de mascotas, monitoreo de estrés, monitoreo de ejercicio. Estas soluciones se tendrán en cuenta para mejoras futuras.



%------------------------------------------------------------------------------%
% \section{Metodología}
% %------------------------------------------------------------------------------%

% Describir usando las metodologías Ágil SCRUM y LEAN cual es la metodología de trabajo que se ha seguido. En él se debe especificar, entre otras cosas, lo siguiente:

% \begin{itemize}
%   \item Los roles dentro de la metodología SCRUM
%   \item ¿Cómo se van a realizar los \textit{sprints}, con que periodicidad?
%   \item Listado inicial de historias
%   \item ¿Qué herramientas se van a usar para la gestión de las historias?
%   \item ¿Cuál es el criterio que se va a seguir para establecer los puntos de esfuerzo?
%   \item Define como se van a medir los resultadosSelección de prototipos y criterio para hacerlo  
% \end{itemize}

% \begin{quotebox}
%   Este apartado corresponde al entregable de la asignatura titulado: \textbf{SCRUM Y LEAN}.
% \end{quotebox}

% %------------------------------------------------------------------------------%
% \section{Implementación de la propuesta}
% %------------------------------------------------------------------------------%

% La implementación debe describir cómo se llevaría a cabo la aplicación de tu propuesta de innovación en el mundo empresarial. Si crees necesario añadir o variar las secciones, puedes hacerlo libremente.

% \subsection{Planificación y estimación}

% Cómo se llevaría a cabo la implementación. Que herramientas, tecnologías, origen de datos, arquitectura software y hardware se necesita.

% \begin{quotebox}
%   Opcionalmente puedes entregar un prototipo de la implementación. En caso de que exista el prototipo deberá estar accesible en repositorio público y deberá incluirse el enlace en la presente sección.
% \end{quotebox}

% \subsection{Despliegue}

% Plan de despliegue del proyecto. 

% \begin{itemize}
%   \item ¿Cómo se va a desplegar?
%   \item Plan de contingencias en el despliegue
%   \item Securización y protección de datos (si fuera necesario)
%   \item Herramientas utilizadas para el despliegue
%   \item Configuración de los diferentes entornos de desarrollo y producción que necesites
% \end{itemize}

% \subsection{Mantenimiento}

% Plan de mantenimiento previsto. 

% \begin{itemize}
%   \item ¿Qué cambios habría que hacer?
%   \item ¿Cuál será el ciclo de vida estimado del proyecto?  
% \end{itemize}


% %------------------------------------------------------------------------------%
% \section{Validación y diseño experimental}
% %------------------------------------------------------------------------------%

% Basándote en la definición de \textit{Minimun Viable Product} y a los criterios de medición que has definido en la sección 4 bajo la metodología \textit{LEAN Startup}, define cual será el criterio final para la validación de tu diseño innovador. 


% %------------------------------------------------------------------------------%
% \section{Conclusiones y trabajo futuro}
% %------------------------------------------------------------------------------%

% En este apartado tendrás que exponer las conclusiones y lineas futuras que de esta propuesta de innovación puedan derivarse.

% Lo siguiente no tiene nada que ver con la estructura, si no con el formato. 

% A continuación, se indica con un ejemplo cómo deben introducirse los títulos y las fuentes en Tablas y Figura. Nota que no se introducen del mismo modo en ambos tipos de recursos.

% Ejemplo de nota al pie\footnote[1]{Ejemplo de nota al pie.}

% Probando \textlf{Hola Mundo}


% \begin{figure}
%   \label{fig:logovertical}
%   \centering
%   \caption[Proceso de percepción de objetos.]{Ejemplo del un árbol de decisión, para decidir qué tipo de carro es el correcto para cada cliente. Cada hoja o nodo representa una variable, las ramas representan el umbral de decisión o la variable a elegir.}
%   \includegraphics[width=0.5\columnwidth]{logovertical.png}
%   \source{Obtenido de \figurecite{NuevoLaredo2021}.}
% \end{figure}

% \begin{table}
%   \label{tab:atributos}
%   \caption{Atributos y técnicas más frecuentemente usados en algunos modelos de predicción de caída de lluvia}
%   \centering
%   \begin{tabular}{p{0.2\tablelength}p{0.2\tablelength}p{0.4\tablelength}p{0.4\tablelength}p{0.8\tablelength}}
%     \toprule
%     Regiones & Périodo & Técnica & Evaluación & Variables predictoras \\
%     \midrule
%     Local, Regional, nacional. 
%     & Anual, Mensual, semanal 
%     & Redes Neuronales, \textit{ARIMA}, Árboles de decisión, Medias móviles, \textit{ABFNN}, $k$-media
%     & RMSE, MAE, MSE, Coeficiente de Pearson 
%     & Cantidad media de lluvia, Mínimo y Máximo de temperatura, Velocidad del viento, latitud y longitud, presión atmósferica, humedad, radación solar, evaporación.\\	  
%     \bottomrule
%   \end{tabular}
%   \source{Versión resumida de \figurecite{poornima2019}.}
% \end{table}


% \begin{listing}[!ht]
%   \label{listing:2}
%   \caption{Hello World in C} 
%   \vspace{-5pt}
%   \begin{minted}{c}
%   #include <stdio.h>
%   int main() {
%     printf("Hello, World!"); /*printf() outputs the quoted string*/
%     return 0;
%   }
%   \end{minted}
%   %\source{Obtenido de \figurecite{poore2021}.}
% \end{listing}

% %------------------------------------------------------------------------------%
% \section{Conclusiones y trabajo futuro} %------------------------------------------------------------------------------%

% Suele empezar con un resumen del problema tratado, de cómo se ha abordado y de por qué la solución sería válida. Es recomendable que incluya también un resumen de las contribuciones del trabajo, en el que relaciones las contribuciones y los resultados obtenidos con los objetivos que habías planteado para el trabajo, discutiendo hasta qué punto has conseguido resolver los objetivos planteados.

% Finalmente, se suele dedicar unos últimos párrafos a hablar de líneas de trabajo futuro que podrían aportar valor añadido al trabajo. La sección debería señalar las perspectivas de futuro que abre el trabajo desarrollado para el campo de estudio definido. En el fondo, debes justificar de qué modo puede emplearse la aportación que has desarrollado y en qué campos.


\pagebreak
%------------------------------------------------------------------------------%
\addcontentsline{toc}{section}{Referencias bibliográficas}
\printbibliography[title=Referencias bibliográficas]
%------------------------------------------------------------------------------%
\pagebreak


%------------------------------------------------------------------------------%
\indexed{section}{Anexo A. Título anexo}



%------------------------------------------------------------------------------%
\end{document}
%%%%%%%%%%%%%%%%%%%%%%%%%%%%%%%%%%%%%%%%%%%%%%%%%%%%%%%%%%%%%%%%%%%%%%%%%%%%%%%%